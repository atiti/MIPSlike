\subsubsection{The ARM architecture}
ARM (Advanced RISC Machine) is a 32-bit reduced instruction set computer (RISC) instruction set architecture (ISA). ARM has evolved over time, with the most recent revision defining three profiles: application, real-time and microcontroller. Each profile adds various instructions, so, in order to keep it simple, the following will consider the original implementation of ARM. Originally ARM was hardwired without microcode in order to keep the design clean. It included the following RISC features:

\begin{itemize}
\item Load/store architecture.
\item No support for misaligned memory accesses.
\item Uniform 32-bit register file.
\item Fixed instruction width of 32 bits.
\item Mostly single-cycle execution.
\end{itemize}

Some additional design features were used to compensate for the simple design:

\begin{itemize}
\item Conditional execution of most instructions, in order to reduce branch overhead and compensate for the lack of a branch predictor.
\item Arithmetic instructions only alter condition codes when desired.
\item 32-bit barrel shifter which can be used without performance penalty with most arithmetic instructions and address calculations.
\item Powerful indexed addressing modes.
\item A link register for fast leaf function calls.
\item Simple, but fast, 2-priority-level interrupt subsystem with switched register banks.
\end{itemize}
Newer revisions now have support for misaligned memory accesses, with some exceptions related to load/store multiple word instructions. ARM has 37 Registers in total, all of which are 32-bits long. These are used as follows:
\begin{itemize}
\item 1 dedicated program counter
\item 1 dedicated current program status register
\item 5 dedicated saved program status registers
\item 30 general purpose registers
\end{itemize}
Earlier ARM implementations have a three stage pipeline, the stages being: fetch, decode and execute. Later designs have deeper pipelines for higher performance. The ARM processor used in iPhone 3GS has 13 stages. Other features and instructions have been developed to enhance performance, particularly the applications profile has been enhanced for multimedia performance.






